%% !Mode:: "TeX:UTF-8"
%% !TEX root = USTB-main.tex
%\iffalse
%\bibliography{reference/reference.bib} % 欺骗latextools获取bib文件
%\fi
%
%%%%%%%% 正文 %%%%%%%
%
%\chapter{总结与展望}
%
%\section{总结}
%
%本研究实现了一种任意维度数据局部连续表示的方法,每份数据被编码为和输入低分辨率数据维度相同的特征图,
%通过一个共享的全连接神经网络,每个点的灰度强度根据它的高分辨率坐标、邻域的特征向量和网格大小计算得到,
%在连续表示的基础上通过离散采样获得任意倍率超分辨率的结果。具体而言:
%
%第一章绪论指明了本研究的来源是解决显微光场成像在分辨率上受限的问题,并且从应用角度分析了高分辨率观测的重要性,
%分析了国内外在光场超分辨率的研究现状以及最新的二维图像超分辨率研究算法,
%明确本研究的目标是实现四维光场角度和空间分辨率任意倍率的超分辨率。
%
%第二章理论依据首先约定了四维光场的符号系统,通过物理成像过程的推导,从理论上说明显微光场任一维度任意倍率超分辨率的可行性,
%从光场重聚焦的推导具体分析了为什么要提高光场数据的分辨率,介绍了超分辨率的量化指标。建立了将离散数据连续恢复的模型,
%并与传统的离散卷积重建相比较,提出连续恢复的优势是分离了重建损失和采样损失。介绍了研究用到的两种神经网络形式。
%
%第三章提出了连续重建的基本框架,建立了连续重建和离散重建之间的关系,给出了使用局部隐函数连续表示光场数据的一个方案。提出了高维数据的编码策略是使用维度相匹配的运算单元,
%完成了高维立方滤波、高维特征编码、坐标采样和隐函数表示等模块的设计。
%
%第四章具体介绍了实验的全部细节以及量化结果和可视化结果,与高维立方插值算法的比较说明了算法的有效性,实现了研究的目标即
%实现光场图像空间分辨率和角度分辨率同时、任意倍率的超分重建。通过实验本研究发现光场图像空间超分辨率可以在细节上使观测更清晰,
%而角度超分辨率则是拓展了显微观测的深度范围。
%
%
%
%\section{展望}
%
%\subsection{任意维度的迁移}
%
%原有超分辨率算法都是针对于某一特定的维度,比如一维语音,二维图像,三维视频,三维体结构,四维光场等,
%而本研究在设计网络实现四维光场超分辨率时,考虑到了其他维度的迁移,比如本研究提到的高维数据编码器、高维数据网格采样、高维坐标、
%高维立方插值等模块,都可以适用到任意维度空间,因此本研究提出的网络具有广泛的适用范围。
%
%由于本研究来源于四维显微光场的超分辨率,因此在有限的时间内本研究只测试了四维超分辨率的性能,
%未来可以测试该模型在其他维度上的表现效果,比如三维体结构或视频的超分辨率、一维语音的增强等。
%
%\subsection{改进之处}
%
%与高维立方插值算法相比,本研究提出的模型在超分辨率预测时速度超过立方插值算法的10倍,需要在时间上进行优化。
%并且高维立方插值算法本质上是一个模型(一组参数),可以迁移到任意维度的数据上并且还能保持数据的特征,
%而本研究提出的模型在面对三维、四维等不同维度的数据时需要训练不同数量的参数,在功能上还略于立方插值算法。
%未来可以将模型进一步改进至对于任意维度的数据,都能使用一组参数进行任一维度任意分辨率的超分重建。
%
%由于使用数据集不同,在有限时间内本研究仅仅与高维立方插值算法做了比较,与当前其他超分辨率算法只做了功能上的比较,
%即本研究提出的超分辨率算法可以实现角度、空间任一维度任意倍率的超分辨率,并且只使用同一个模型,
%超过之前只能在固定角度倍率训练某一倍率的空间超分辨率模型,固定空间倍率训练某一倍率的角度超分辨率模型的算法架构,
%然而并没有与这些光场超分辨率算法在性能上做比较。
%
%目前仅使用峰值信噪比作为唯一评价指标,而没有使用结构相似性是因为计算高维数据的二维切片之间的结构相似性并加权求和的方式
%在本研究看来是不合理的,后续研究将实现结构相似性在高维空间中的计算方式。
%
%在训练好的模型迁移到其他类型的光场数据上时,会出现轻微的伪影,这也是需要改进的地方。
%
%\subsection{发现的新问题}
%
%在本研究提出的范式中,数据的表示本质上是通过局部隐函数,而局部隐函数的一个问题是,
%虽然对任意精细的坐标可以查询到函数值,然而无法在函数维度上进行操作,比如对函数的微积分、傅里叶变换等操作。
%实际应用可能会用到的函数维度上的操作,比如式(2-3)的积分操作,隐函数表示还是只能通过求和的方式实现极限的逼近,
%那么是不是可以通过神经网络学习到一种连续表示,这种连续表示可以实现诸如微积分、傅里叶变换这类函数级别的操作呢?
%
%本研究提出的超分辨率重建算法本质上用到了低分辨率图像和高分辨率图像的局部相关性,
%即如果把同一份数据的不同分辨率放在同一个坐标系下,它们之间的坐标对应关系可以很容易建立起来。
%而在有些领域输入和输出之间的坐标对应关系不是很容易建立,比如三维重建,此时输入数据和输出数据甚至没有相同的维度数量,
%也就是无法放在同一坐标系下建模,跨坐标系下的连续表示的恢复也是一个未来的研究方向。
